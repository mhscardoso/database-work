\section{Curso Iniciante}
Queries SQL realizadas para o curso SQL Course online. Estas, para a parte inicial do mesmo curso.

\subsection{Selecting Data}

1.) - Exiba o primeiro nome e a idade de todos os que estão na tabela.

\begin{lstlisting}
    SELECT first, age FROM empinfo;
\end{lstlisting}

\begin{tabular}{l l}
    {\textbf{first}} & {\textbf{age}} \\
    {John}           & {45}           \\
    {Mary}           & {25}           \\
    {Eric}           & {32}           \\
    {Mary Ann}       & {32}           \\
    {Ginger}         & {42}           \\
    {Sebastian}      & {23}           \\
    {Gus}            & {35}           \\
    {Mary Ann}       & {52}           \\
    {Erica}          & {60}           \\
    {Leroy}          & {22}           \\
    {Elroy}          & {22}          
\end{tabular}\\ \\

2.) - Exiba o nome, o sobrenome e a cidade de todos que não são de Payson.

\begin{lstlisting}
    SELECT first, last, city FROM empinfo WHERE city <> "Payson";
\end{lstlisting}

\begin{tabular}{l l l}
    {\textbf{first}} & {\textbf{last}} & {\textbf{city}} \\
    {Eric}           & {Edwards}       & {San Diego}     \\
    {Mary Ann}       & {Edwards}       & {Phoenix}       \\
    {Ginger}         & {Howell}        & {Cottonwood}    \\
    {Sebastian}      & {Smith}         & {Gila Bend}     \\
    {Gus}            & {Gray}          & {Bagdad}        \\
    {Mary Ann}       & {May}           & {Tucson}        \\
    {Erica}          & {Williams}      & {Show Low}      \\
    {Leroy}          & {Brown}         & {Pinetop}       \\
    {Elroy}          & {Cleaver}       & {Globe}        
\end{tabular} \\ \\

3.) - Exiba todas as colunas para todas as pessoas com mais de 40 anos.

\begin{lstlisting}
    SELECT * FROM empinfo WHERE age > 40;
\end{lstlisting}

\begin{tabular}{l l l l l l}
{\textbf{first}} & {\textbf{last}} & {\textbf{id}} & {\textbf{age}} & {\textbf{city}} & {\textbf{state}} \\
{John}           & {Jones}         & {99980}       & {45}           & {Payson}        & {Arizona}        \\
{Ginger}         & {Howell}        & {98002}       & {42}           & {Cottonwood}    & {Arizona}        \\
{Mary Ann}       & {May}           & {32326}       & {52}           & {Tucson}        & {Arizona}        \\
{Erica}          & {Williams}      & {32327}       & {60}           & {Show Low}      & {Arizona}       
\end{tabular} \\ \\ 

4.) - Exiba o nome e o sobrenome de todos cujo sobrenome termina em “ay”.

\begin{lstlisting}
    SELECT first, last FROM empinfo WHERE last LIKE "%ay"
\end{lstlisting}

\begin{tabular}{l l l}
    {\textbf{first}} & {\textbf{last}}\\
    {Gus}            & {Gray}         \\
    {Mary Ann}       & {May}
\end{tabular} \\ \\

5.) - Exiba todas as colunas para todos cujo primeiro nome é igual a “Mary”.

\begin{lstlisting}
    SELECT * FROM empinfo WHERE first = "Mary";
\end{lstlisting}

\begin{tabular}{l l l l l l}
    {\textbf{first}} & {\textbf{last}} & {\textbf{id}} & {\textbf{age}} & {\textbf{city}} & {\textbf{state}} \\
    {Mary}           & {Jones}         & {99982}       & {25}           & {Payson}        & {Arizona}       
\end{tabular} \\ \\ 

6.) - Exiba todas as colunas para todos cujo primeiro nome contenha “Mary”.

\begin{lstlisting}
    SELECT * FROM empinfo WHERE first LIKE "%Mary%";
\end{lstlisting}

\begin{tabular}{llllll}
    {\textbf{first}} & {\textbf{last}} & {\textbf{id}} & {\textbf{age}} & {\textbf{city}} & {\textbf{state}} \\
    {Mary}      & {Jones}    & {99982}    & {25}    & {Payson}   & {Arizona}        \\
    Mary Ann    & Edwards    & 88233      & 32      & Phoenix    & Arizona          \\
    Mary Ann    & May        & 32326      & 52      & Tucson     & Arizona                              
\end{tabular}


\subsection{Creating Tables}

\begin{lstlisting}
    CREATE TABLE myemployees(
        firstname  VARCHAR(50),
        lastname   VARCHAR(50),
        title      VARCHAR(50),
        age        NUMBER(128),
        salary     NUMBER(2000000)
    );
\end{lstlisting}


\subsection{Inserting Into a Table}

\begin{lstlisting}
    INSERT INTO 
    myemployees
        (firstname, lastname, title, age, salary)
    VALUES 
        ("Jonie", "Weber", "Secretary", 28, 19500.00);

    INSERT INTO 
    myemployees
        (firstname, lastname, title, age, salary)
    VALUES 
        ("Potsy", "Weber", "Programmer", 32, 45300.00);
    
    INSERT INTO 
    myemployees
        (firstname, lastname, title, age, salary)
    VALUES 
        ("Dirk", "Smith", "Programmer II", 45, 75020.00);
\end{lstlisting}

1.) - Selecione todas as colunas para todos em sua tabela de funcionários.

\begin{lstlisting}
    SELECT * FROM myemployees;
\end{lstlisting}

\begin{tabular}{l l l l l}
    {\textbf{firstname}} & {\textbf{lastname}} & {\textbf{title}} & {\textbf{age}} & {\textbf{salary}} \\
    {Jonie}              & {Weber}             & {Secretary}      & {28}           & {19500}           \\
    {Potsy}              & {Weber}             & {Programmer}     & {32}           & {45300}           \\
    {Dirk}               & {Smith}             & {Programmer II}  & {45}           & {75020}          
\end{tabular} \\ \\

2.) - Selecione todas as colunas para todos com um salário acima de 30.000.

\begin{lstlisting}
    SELECT * FROM myemployees WHERE salary > 30000;
\end{lstlisting}

\begin{tabular}{l l l l l}
    {\textbf{firstname}} & {\textbf{lastname}} & {\textbf{title}} & {\textbf{age}} & {\textbf{salary}} \\
    {Potsy}              & {Weber}             & {Programmer}     & {32}           & {45300}           \\
    {Dirk}               & {Smith}             & {Programmer II}  & {45}           & {75020}          
\end{tabular} \\ \\

3.) - Selecione o nome e o sobrenome de todas as pessoas com menos de 30 anos.

\begin{lstlisting}
    SELECT firstname, lastname FROM myemployees WHERE age < 30;
\end{lstlisting}

\begin{tabular}{l l}
    {\textbf{firstname}} & {\textbf{lastname}} \\
    {Jonie}              & {Weber}
\end{tabular} \\ \\

4.) - Selecione o nome, o sobrenome e o salário de qualquer pessoa com “Programmer” no cargo.

\begin{lstlisting}
    SELECT firstname, lastname FROM myemployees WHERE title LIKE "%Programmer%";
\end{lstlisting}

\begin{tabular}{l l}
    {\textbf{firstname}} & {\textbf{lastname}} \\
    {Potsy}              & {Weber}             \\
    {Dirk}               & {Smith}             
\end{tabular} \\ \\

5.) - Selecione todas as colunas para todos cujo sobrenome contenha “ebe”.

\begin{lstlisting}
    SELECT * FROM myemployees WHERE lastname LIKE "%ebe%";
\end{lstlisting}

\begin{tabular}{l l l l l}
    {\textbf{firstname}} & {\textbf{lastname}} & {\textbf{title}} & {\textbf{age}} & {\textbf{salary}} \\
    {Jonie}              & {Weber}             & {Secretary}      & {28}           & {19500}           \\
    {Potsy}              & {Weber}             & {Programmer}     & {32}           & {45300}
\end{tabular} \\ \\

6.) - Selecione o primeiro nome para todos cujo primeiro nome é igual a “Potsy”.

\begin{lstlisting}
    SELECT firstname FROM myemployees WHERE firstname = "Potsy";
\end{lstlisting}

\begin{tabular}{l}
    {\textbf{firstname}} \\
    {Potsy}          
\end{tabular} \\ \\

7.) - Selecione todas as colunas para todos com mais de 80 anos.

\begin{lstlisting}
    SELECT * FROM myemployees WHERE age > 80;
\end{lstlisting}

\begin{tabular}{l l l l l}
    {\textbf{firstname}} & {\textbf{lastname}} & {\textbf{title}} & {\textbf{age}} & {\textbf{salary}}
\end{tabular} \\ \\

8.) - Selecione todas as colunas para todos cujo sobrenome termina em “ith”

\begin{lstlisting}
    SELECT * FROM myemployees WHERE lastname LIKE "%ith";
\end{lstlisting}

\begin{tabular}{l l l l l}
    {\textbf{firstname}} & {\textbf{lastname}} & {\textbf{title}} & {\textbf{age}} & {\textbf{salary}} \\
    {Dirk}               & {Smith}             & {Programmer II}  & {45}           & {75020}          
\end{tabular} \\ \\


\subsection{Updating Records}

1.) - Jonie Weber acabou de se casar com Bob Williams. Ela solicitou que seu sobrenome fosse atualizado para Weber-Williams.

\begin{lstlisting}
    UPDATE myemployees 
        SET lastname = "Weber-Williams" 
        WHERE firstname = "Jonie";

    SELECT * FROM myemployees;
\end{lstlisting}

\begin{tabular}{l l l l l}
    {\textbf{firstname}} & {\textbf{lastname}} & {\textbf{title}} & {\textbf{age}} & {\textbf{salary}} \\
    {Jonie}              & {Weber-Williams}    & {Secretary}      & {28}           & {19500}           \\
    {Potsy}              & {Weber}             & {Programmer}     & {32}           & {45300}           \\
    {Dirk}               & {Smith}             & {Programmer II}  & {45}           & {75020}          
\end{tabular} \\ \\ 

2.) - O aniversário de Dirk Smith é hoje, adicione 1 a sua idade.

\begin{lstlisting}
    UPDATE myemployees 
	    SET age = age + 1 
	    WHERE firstname = "Dirk";

    SELECT * FROM myemployees;
\end{lstlisting}

\begin{tabular}{l l l l l}
    {\textbf{firstname}} & {\textbf{lastname}} & {\textbf{title}} & {\textbf{age}} & {\textbf{salary}} \\
    {Jonie}              & {Weber-Williams}    & {Secretary}      & {28}           & {19500}           \\
    {Potsy}              & {Weber}             & {Programmer}     & {32}           & {45300}           \\
    {Dirk}               & {Smith}             & {Programmer II}  & {46}           & {75020}          
\end{tabular} \\ \\

3.) - Todas as secretárias passam a ser chamadas de “Auxiliar Administrativo”. Atualize todos os títulos de acordo.

\begin{lstlisting}
    UPDATE myemployees 
	    SET title = "Administrative Assistant" 
	    WHERE title = "Secretary";

    SELECT * FROM myemployees;
\end{lstlisting}

\begin{tabular}{l l l l l}
    {\textbf{firstname}} & {\textbf{lastname}} & {\textbf{title}} & {\textbf{age}} & {\textbf{salary}} \\
    {Jonie}              & {Weber-Williams}    & {Administrative Assistant}      & {28}           & {19500}           \\
    {Potsy}              & {Weber}             & {Programmer}     & {32}           & {45300}           \\
    {Dirk}               & {Smith}             & {Programmer II}  & {46}           & {75020}          
\end{tabular} \\ \\

4.) - Todos que estão ganhando menos de 30.000 receberão um aumento de 3.500 por ano.

\begin{lstlisting}
    UPDATE myemployees 
	    SET salary = salary + 3500 
	    WHERE salary < 30000;

    SELECT * FROM myemployees;
\end{lstlisting}

\begin{tabular}{l l l l l}
    {\textbf{firstname}} & {\textbf{lastname}} & {\textbf{title}} & {\textbf{age}} & {\textbf{salary}} \\
    {Jonie}              & {Weber-Williams}    & {Administrative Assistant}      & {28}           & {23000}           \\
    {Potsy}              & {Weber}             & {Programmer}     & {32}           & {45300}           \\
    {Dirk}               & {Smith}             & {Programmer II}  & {46}           & {75020}          
\end{tabular} \\ \\

5.) - Todos que estão ganhando mais de 33.500 receberão um aumento de 4.500 por ano.

\begin{lstlisting}
    UPDATE myemployees 
	    SET salary = salary + 4500 
	    WHERE salary > 33500;

    SELECT * FROM myemployees;
\end{lstlisting}

\begin{tabular}{l l l l l}
    {\textbf{firstname}} & {\textbf{lastname}} & {\textbf{title}} & {\textbf{age}} & {\textbf{salary}} \\
    {Jonie}              & {Weber-Williams}    & {Administrative Assistant}      & {28}           & {23000}           \\
    {Potsy}              & {Weber}             & {Programmer}     & {32}           & {49800}           \\
    {Dirk}               & {Smith}             & {Programmer II}  & {46}           & {79520}          
\end{tabular} \\ \\

6.) - Todos os títulos de “Programador II” agora são promovidos a “Programador III”

\begin{lstlisting}
    UPDATE myemployees 
	    SET title = "Programmer III" 
	    WHERE title = "Programmer II" ;

    SELECT * FROM myemployees;
\end{lstlisting}

\begin{tabular}{l l l l l}
    {\textbf{firstname}} & {\textbf{lastname}} & {\textbf{title}} & {\textbf{age}} & {\textbf{salary}} \\
    {Jonie}              & {Weber-Williams}    & {Administrative Assistant}      & {28}           & {23000}           \\
    {Potsy}              & {Weber}             & {Programmer}     & {32}           & {49800}           \\
    {Dirk}               & {Smith}             & {Programmer III}  & {46}           & {79520}          
\end{tabular} \\ \\

7.) - Todos os títulos de “Programador” agora são promovidos a “Programador II”

\begin{lstlisting}
    UPDATE myemployees 
	    SET title = "Programmer II" 
	    WHERE title = "Programmer" ;

    SELECT * FROM myemployees;
\end{lstlisting}

\begin{tabular}{l l l l l}
    {\textbf{firstname}} & {\textbf{lastname}} & {\textbf{title}} & {\textbf{age}} & {\textbf{salary}} \\
    {Jonie}              & {Weber-Williams}    & {Administrative Assistant}      & {28}           & {23000}           \\
    {Potsy}              & {Weber}             & {Programmer II}     & {32}           & {49800}           \\
    {Dirk}               & {Smith}             & {Programmer III}  & {46}           & {79520}          
\end{tabular} \\ \\


\subsection{Deleting Records}

1.) - Jonie Weber-Williams acabou de sair, remova seu registro da tabela.

\begin{lstlisting}
    DELETE FROM myemployees
	    WHERE firstname = "Jonie" AND lastname = "Weber-Williams";

    SELECT * FROM myemployees;
\end{lstlisting}

\begin{tabular}{l l l l l}
    {\textbf{firstname}} & {\textbf{lastname}} & {\textbf{title}} & {\textbf{age}} & {\textbf{salary}} \\
    {Potsy}              & {Weber}             & {Programmer II}     & {32}           & {49800}           \\
    {Dirk}               & {Smith}             & {Programmer III}  & {46}           & {79520}          
\end{tabular} \\ \\

2.) - É hora de cortes orçamentários. Remova todos os funcionários que estão ganhando mais de 70.000 dólares.

\begin{lstlisting}
    DELETE FROM myemployees
	    WHERE salary > 70000;

    SELECT * FROM myemployees;
\end{lstlisting}

\begin{tabular}{l l l l l}
    {\textbf{firstname}} & {\textbf{lastname}} & {\textbf{title}} & {\textbf{age}} & {\textbf{salary}} \\
    {Potsy}              & {Weber}             & {Programmer II}     & {32}           & {49800}       
\end{tabular} \\ \\

\subsection{Drop a Table}

\begin{lstlisting}
    DROP TABLE myemployees;
\end{lstlisting}